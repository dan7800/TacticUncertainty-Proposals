\documentclass[11pt]{article}
\usepackage{fullpage}
\usepackage{times}
\usepackage{url}
\usepackage{color} % needed for todo
\usepackage{enumitem}

\newcommand{\todo}[1]{\textcolor{cyan}{\textbf{[#1]}}}
\newcommand{\dan}[1]{\textcolor{blue}{{\it [Dan says: #1]}}}


\newcommand {\doublespace} {\addtolength{\baselineskip}{.5\baselineskip}}
\newcommand {\singlespace} {\addtolength{\baselineskip}{-.333\baselineskip}}

\begin{document}
%\pagestyle{empty}
\renewcommand{\thepage}{DMP-\arabic{page}}
\setcounter{page}{1}

\def\myskip{3ex}

\centerline{\normalsize Rochester Institute of Technology}
\vspace{4 mm}
\centerline{\normalsize\bf Addressing Decision-Making Uncertainty in Cyber-physical Systems By Accounting for Tactic Volatility}


\section*{Data Management Plan}
%This plan articulates how sharing of the primary data generated during
%the course of the project for this proposal will be implemented.

\subsection*{Expected Data}

The expected data to be generated during the course of this project are:

\begin{enumerate}[noitemsep]

  \item Evaluation metrics of created machine learning model in simulations and implementation in CPS.
  \item Data created by CPS including, but not limited to: Tactic uncertainty, tactic volatility, and tactic attributes.
  \item Technical papers describing our findings.
  \item Data from implementation into CPS.
  \item Student participation in outreach activity.
  \item Outreach event observations and student feedback.
  \item Educational material from outreach events.
  
\end{enumerate}

\subsection*{Data Retention and Storage}
All generated data from evaluations and simulations will be made publicly available on the project website. Project findings will be disseminated via publications in conference papers, in journals and through workshops. To protect student privacy, the project management team will only share the student outreach data in summary form so that we can monitor the level of dissemination for any required reporting. Confidential data, including individual survey responses will not be released to outside parties. This data will be retained throughout the life of the project in anonymous form and used to generate annual reports and the final report to NSF. Once the project has concluded, this data will be retained for one year, then purged. All non-confidential data generated by this project will be retained for a minimum of three years after the conclusion of the award, including data that is not specifically disseminated as described in the following section. The PI will use file storage services provided by RIT to ensure secure preservation of data during the retention period. Syracuse University will manage their data regarding in a similar manner as RIT.

% \vspace{3mm}
% \noindent \textbf{Syracuse University Data Plan}




\subsection*{Data Dissemination}

Technical papers generated during the course of the project will be published in academic journals and conference proceedings. Papers published in venues without archival proceedings, as well as technical reports not otherwise published, will be disseminated via arXiv.

%Related papers and/or conference presentations submitted by the PIs will be made publicly available if allowed by the accepting institution or organization. If the institution or organization accepting the paper or presentation assumes the copyright, we will post a full citation so interested parties can locate the published document.

Data produced from the proposed research activities will be made available either publicly on the project reporting website or, for sensitive data, available upon request by other researchers. All created project materials will be made available through the project's website. Standard reports for the NSF will be generated and electronically submitted through normal channels. The management team may use summary reports for pedagogical purposes so that other approved institutions may see the results and lessons learned. Annual and final reports will evaluate the adherence to this Data Management Plan. %Shared data will be stored in standard formats, such as text files and XML, whenever it is possible to do so. If binary formats are necessary (e.g., for saving space), utilities for converting this data into human-readable formats will be made available.



\end{document}
