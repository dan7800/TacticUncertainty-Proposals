\documentclass[12pt]{article}

\usepackage{cite}
\usepackage{listings}
\usepackage{times}
\usepackage{color}
\usepackage{url}
\usepackage{multirow}
\usepackage{multicol}
\usepackage{enumitem} % Use for enumerating A, B, C etc...
\urlstyle{same} % Used for formatting formatting url footnotes
\usepackage{fancyhdr} % Header
%\usepackage[table]{xcolor}% http://ctan.org/pkg/xcolor
\usepackage[table,xcdraw]{xcolor} % helps to format TOC
\usepackage{soul} % highlighting
%\usepackage{pgfgantt} % Project timeline
\usepackage[titletoc,toc,title]{appendix} % Need for appendix, page numbering
\usepackage{tikz}
\usetikzlibrary{calc,arrows.meta,fit,positioning}
%\usepackage{amssymb,graphicx} % Events and milestones
\usepackage{amsmath}
\usepackage{mathtools}
%\usepackage{bbm}
\usepackage{amssymb}
\usepackage{lastpage}

%\usepackage{algorithm}
%\usepackage[noend]{algpseudocode}


\usepackage{xspace} % Needed for et al.
\newcommand{\ie}{\emph{i.e.,}\xspace}
\newcommand{\eg}{\emph{e.g.,}\xspace}
\newcommand{\etc}{etc.\xspace}
\newcommand{\etal}{\emph{et~al.}\xspace}  

% Page margins etc....
\usepackage[bottom=1in, left=1in, right=1in, top=1in]{geometry} % Should all be 1
%.85

\usepackage[english]{babel}
\usepackage[utf8x]{inputenc}
\usepackage{graphicx}
%\usepackage{wrapfig}
%\usepackage{lipsum}
%\usepackage{pgfgantt}

%\usepackage{minted} % Side by side code

\newcommand{\todo}[1]{\textcolor{cyan}{\textbf{[#1]}}}
\newcommand{\dan}[1]{\textcolor{blue}{{\it [Dan: #1]}}}
\newcommand{\qi}[1]{\textcolor{red}{{\it [Qi: #1]}}}
\newcommand{\alex}[1]{\textcolor{green}{{\it [Alex: #1]}}}
\newcommand{\travis}[1]{\textcolor{brown}{{\it [Travis: #1]}}}

%%% Start column formatting
% Note: In overleaf sometimes columns fail to render. Check on PDF Output
\newcolumntype{L}[1]{>{\raggedright\arraybackslash}m{#1}} % raggedright= align left
\definecolor{Gray}{gray}{0.80} % the lower the #, the darker it gets
%%% End Table formatting

%%%% Start toggling showing/hiding some information
\newif\ifShowAll
\ShowAlltrue % Display All Info
%\ShowAllfalse % Hide Some Info


% Using Recurrent Neural Networks to Account for Tactic Volatility and Improve the Decision-Making Process of Autonomous Systems 

\newcommand{\Title}{Accounting for Tactic Volatility: Improving Autonomous System Decision-Making Using Recurrent Neural Networks}

\newcommand{\shortTitle}{Accounting for Tactic Volatility in Autonomous Systems} % Used in the heading just so it fits

\newcommand{\CallNumber}{N00421-18-S-0001} % BAA Number
\newcommand{\CallName}{XXXX}
%\newcommand{\BAANumber}{N00421-18-S-0001}

\usepackage{fancyhdr} % Header
\pagestyle{fancy}
\lhead{\emph{\shortTitle}}
%\rhead{Krutz (RIT)}
\rhead{Krutz: dxkvse@rit.edu}
%\lhead{\shortTitle}
%\rhead{}


%\setlength\cftparskip{-.7pt} %% Table of contents spacing
%\setlength\cftbeforechapskip{0pt}

\begin{document}

\begin{titlepage}

\newcommand{\HRule}{\rule{\linewidth}{0.3mm}} % Defines a new command for the horizontal lines, change thickness here

%%%%%%% Start new Title format


%% DK: I am not sure if we should have this
%\noindent\large \CallName, \CallNumber\\[.20cm] % Call Name

%  \textsc{\Large White Paper Submission\dan{update page with required information}}\\[0.5cm] % Major heading

%\noindent\large \CallNumber\\[.20cm] % Call Name


\begin{center}
  \textsc{\Large White Paper Submission}\\[0.5cm] % Major heading such as course name
  \textsc{\large \CallNumber}\\[1.5cm]   %% Fix this
\end{center}

%\noindent \LARGE \textbf{\Title}\\[.10cm] % Title
\noindent \Large \textbf{\Title}\\[.10cm] 


\noindent \large  \underline{\textbf{Technical Proposal}}\\ [.15cm] 

\begin{tabular}{ L{50mm} L{100mm} }

%\normalsize \textbf{Technical Proposal:} & \normalsize  \CallName, \CallNumber  \\

%\noindent\large Technical Proposal: N00174-18-0001\\[.20cm]
%\noindent\large NEC Technical POC: \\[.20cm]
%\noindent\large Topic Number: \\[.20cm]

%\normalsize \textbf{BAA Number:} & \normalsize \CallNumber  \\
%\normalsize \textbf{Proposed Title:} & \normalsize  \Title  \\ 
\normalsize \textbf{Research opportunity area:} & \normalsize  Research and Intelligence  \\ 
% Research opportunity area of interest

\normalsize \textbf{Technical POC/PI:} & \normalsize  Dr. Daniel Krutz \\
 & \vspace{-2mm} \normalsize Department of Software Engineering
 \\

   & \vspace{-4mm} \normalsize 152 Lomb Memorial Drive \\
   & \vspace{-6mm} \normalsize Rochester, NY 14623 \\
   & \vspace{-8mm} \normalsize Phone: (585) 475-2896 \\
   & \vspace{-10mm} \normalsize Email: dxkvse@rit.edu \\

\vspace{-6mm}\normalsize \textbf{Administrative POC:} & \vspace{-6mm} \normalsize Ms. Laura Kleiman \\

%   & \vspace{-8mm} \normalsize Senior Research Administrator  \\
   & \vspace{-9mm} \normalsize Sponsored Research Services  \\
   & \vspace{-11mm} \normalsize Rochester Institute of Technology  \\
   & \vspace{-13mm} \normalsize University Services Center, Suite 2400  \\
   & \vspace{-15mm} \normalsize 141 Lomb Memorial Drive, Rochester, NY 14623-5608   \\
   & \vspace{-17mm} \normalsize Rochester, NY 14623-5608  \\
   & \vspace{-19mm} \normalsize Phone: (585)-475-2262  \\
%   & \vspace{-22mm} \normalsize Facsimile: (585)-475-2262  \\
   & \vspace{-21mm} \normalsize Email: ljksrs@rit.edu  \\


%BAA Number, proposed title, research opportunity area of interest, contracts and technical points of contact, telephone number, facsimile number, and E-mail address


\end{tabular}

 \end{titlepage}

\cfoot{\thepage}
\pagenumbering{alph} % Start roman numbering
\setcounter{tocdepth}{1} % Show sections

\cfoot{} % Leave blank


%%%%% TOC - Start
%\renewcommand\contentsname{Table of Contents}
%\tableofcontents
%\listoffigures
%\listoftables
%\newpage

%%%%% TOC - End


\setcounter{page}{1}
\pagenumbering{arabic} % Switch to normal numbers

%\cfoot{\thepage\ of \pageref{LastPage}}
\fancyfoot[C]{Page~\thepage~of~\pageref{lastpage}}

\section{Technical Concept}

The proposed work will enable autonomous systems to \textbf{better account for uncertainty and volatility} as well as \textbf{operate in real-world environments with limited data}.
%By enabling the system to more accurately account for tactic volatility, this innovative work will {\bf increase the system's resiliency, efficiency, and ability to complete mission critical operations.} This will be accomplished through the use of \emph{Multiple Regression Analysis} (MRA) to estimate tactic volatility and \emph{Autoregressive Integrated Moving Average} (ARIMA) to perform time series forecasting allowing the system to proactively maintain specifications. 

\vspace{1mm}
\emph{Tactics} are actions taken by an autonomous system to deal with environmental changes or to achieve objectives. Example tactics include reducing non-essential functionality of a UAV when battery levels are low or provisioning an additional virtual machine in a web farm when workload reaches a specific threshold. Tactics will frequently experience volatility in terms of latency, cost, dependability, and availability. Unfortunately, state-of-the-art decision-making processes do not account for uncertainty and variability in the attributes of the tactic~\cite{moreno2017adaptation, Moreno:2017:CMP:3105503.3105511, camara2016analyzing}. This inability to account for volatility can lead to inaccurate utility estimates and result in decisions leading to less than optimal outcomes. Therefore, a robust prediction mechanism is needed allow autonomous systems to anticipate and account for tactic volatility and reduce associated uncertainty. \ul{This pioneering research will develop a \textbf{\emph{Tactic Volatility Aware} (TVA)} solution that will enable autonomous systems to better account for volatility using limited quantities of historical data, especially when operating in previously-unseen, stochastic environments.} %This effort is the first that enables systems to account for tactic volatility that will almost invariably/surely be encountered in real-world systems. 
The proposed TVA process will address the following challenges: %and limitations currently faced by autonomous systems: 

\begin{enumerate}[noitemsep]

	\item The proposed process must enable the autonomous system to effectively account for uncertainty and variability in its operating environment. 

	\item Many autonomous systems operate in new, stochastic environments. As a result, the proposed framework will enable the system to make accurate predictions with minimal amounts of historical or prior mission data. TVA will predict not only values but also provide a measure of certainty. In some circumstances, knowing the amount of volatility will be more important than having a specific, predicted value.

	\item In the event that prior mission data is available, the system should be able to quickly and effectively utilize this information to improve its future predictive ability.

    % 	\item Systems must proactively ensure adherence to \emph{Service Level Agreement} (SLA) specifications.

%	\item Systems must proactively ensure it adheres to \emph{Service Level Agreement} (SLA) specifications.

	\item Systems should proactively adhere to \emph{Service Level Agreement} (SLA) specifications.

	\item To support its inclusion into a wide range of existing systems, the proposed framework should easily integrate into existing autonomous decision-making processes and strategies.% such as including MAPE-K~\cite{kephart2003vision} and Rainbow~\cite{garlan2004rainbow}.

\end{enumerate}


%\subsection{Motivating Example}
%\label{sec: motivatingexample}

%\vspace{1mm} 
\noindent\textbf{Motivating Example:} The goal of a cloud-based, self-adaptive system~\cite{moreno2017adaptation} is to maximize utility while minimizing cost. An SLA defines the target response time ($T$) and how utility ($U$) is calculated. %The system incurs penalties if the target response time is not met and accrues rewards for meeting the target average response time against the measurement interval. %In this scenario, the cost is directly proportional to the number of servers used. 
The average response rate is $a$, the average response time is $r$, the maximum request is $k$, and the length of each interval is defined as $\tau$. Provided content is reduced as necessary using a dimmer value, ($d$). Optional content has reward of ($R_O$) which produces a higher reward than mandatory content ($R_M$). Cost ($C$) could be the monetary or energy cost:

\begin{equation} \label{eq: motivatingExample}
	U = \left\{ \begin{array}{rl}
 	(\tau a (dR_O+(1-d)R_M)/C~~~~r\leq T \\ 
  	(\tau \min (0,a-k)R_O)/C~~~~r> T 
       \end{array} \right.
\end{equation}

This system can account for increases in user traffic using two tactic options: (I) reducing the proportion of responses that include the optional content (dimmer), and (II) adding a new server. While reducing optional content has negligible latency, adding a new server can take several minutes. Keeping the system/process presented above in mind, we will next briefly describe the types of volatility that would be encountered and why each is important. % DK: ? Need a transitioning sentence?  <-- ALEX: added transition

\vspace{1mm} \noindent \textbf{Tactic Latency Volatility: }%If the system anticipates that the response threshold will be surpassed in the immediate future, then the system could proactively start the tactic to add a server and keep the response time under the defined threshold. %Alternately, the system may determine that latency readings have become highly volatile while not yet surpassing a threshold, in this case the system may utilize strategies on which latency has less effect. 
%Overestimating latency could result in scenarios where the system unnecessarily incurs additional cost, as servers would be `active' longer than necessary. Additionally, if the system determines that it is likely to surpass the defined response time threshold before a new server can be added, then the most appropriate action may be to use the faster tactic that reduces the amount of optional content while a new server is being added. 
Improper tactic latency predictions can result in the system executing the tactic too soon or too late. It can also lead to the selection of an inappropriate tactic leading to a less than optimal outcome. \emph{Accounting for tactic latency volatility is a paramount concern, especially when utilizing a proactive adaptation approach, or when utilizing complementary tactics.} 
% DK: I don't love the takeaway messages in each of these examples <-- ALEX: took a stab at rewriting takeaways, better?

%\emph{An adaptive, intelligent system should be able to consider more than one tactic and estimate the confidence on each action to allow for a mixture of tactics and more importantly, to facilitate dynamic error-correction. %The system should be able to recognize it made a mistake as quickly as possible and readjust accordingly.}

\vspace{1mm} \noindent \textbf{Tactic Cost Volatility:} It is important to account for cost volatility to ensure accurate utility calculations. In the event that the cost is defined to be higher than what the system is actually encountering in the real-world, then this may lead to scenarios where optional ($O$) content is shown too infrequently. Conversely, if the cost is defined to be lower than what is actually being encountered, then could lead to scenarios where optional ($O$) content is shown too frequently. \emph{A volatility-aware solution that enables a more accurate estimate of the actual cost is needed to enable the system to make better decisions that lead to more optimal outcomes.}

\vspace{1mm} \noindent \textbf{Reactive Specifications Monitoring:} If the motivating example was a purely responsive system and did not employ any proactive functionality, it will only determine if the defined response threshold ($T$) is being surpassed at a given moment. This is problematic since the tactic of adding additional resources to reduce response time in the example entails latency, meaning that the system will need to adapt \emph{before} the tactic is actually needed. Otherwise, the system will incur penalties or not realize rewards while it waits for the tactic to become available. \emph{A process that enables the system to better anticipate how it will perform in accordance with an SLA would help the system to operate optimally in dynamic environments.} %This might include integrating mechanisms for long-term planning and sampling of future, possible trajectories conditioned on a particular SLA.\dan{check the writing of this}}

\vspace{-2mm}
\subsection{Technical Objectives}
TVA will be the first technique to predict and account for tactic volatility during the decision-making process. The key objectives of our work are are:


%Our TVA approach will be the first of its kind to provide support for predicting and directly accounting for tactic volatility in autonomous systems. The key objectives are:

%The proposed tactic volatility prediction framework based upon dynamic Bayesian Matrix Factorization is the first effort to provide support for predicting tactic volatility in self-adaptive/autonomous systems. %\dan{Qi: Can you finish this part}

%\vspace{1mm} \noindent \textbf{Objective I: Predicting Tactical Data and Volatility with Recurrent Neural Networks:} Autonomous systems typically collect
%, at varying frequencies, 
%streams of data through an array of sensors. 

%\vspace{1mm} \noindent \textbf{Objective I: Predicting Tactical Data and Volatility with Recurrent Neural Networks: }Autonomous systems typically collect data via various sensors which arrive as time series with varying frequencies. %Typically this data consists of continuous parameters (\eg altitude, energy consumption, speed, engine RPMs, outsite air temperature, \etc), but may also consist of discrete parameters (\eg weather conditions, user/system commands, \etc). 
%Recurrent Neural Networks (RNNs), especially with modern memory cell architectures -- such as LSTM (long short-term memory), GRU (gated recurrent unit), Delta-RNN and others -- are excellent at determining long time range dependencies and predicting future time series values. However, to more appropriately handle tactic volatility, we will extend these to not only predict future sensor parameters \emph{but also a confidence interval for those predictions} to better inform the system of the risk for using such a prediction.

% We will develop a dynamic Bayesian matrix factorization based prediction model to automatically and precisely quantify the tactic volatility in multiple key aspects including latency, dependability, and availability. By learning from the historical usage data of an autonomous system that covers both the characteristics of its internal components and control algorithms as well as features from the surrounding environment, the prediction model can anticipate the autonomous system's behavior in the near future. The model will also enable the system to anticipate decisions that may lead to undesirable outcomes or lead to erratic system behavior. 

\vspace{1mm} \noindent \textbf{Objective I: Predicting Tactical Data and Volatility with Recurrent Neural Networks: }TVA will utilize a RNN-based solution to predict the volatility of a tactic. This predicted tactic value will be included into the decision-making process to result in more accurate, informed decisions.

\vspace{1mm} \noindent \textbf{Objective II: Dynamic Transfer Learning Using Evolved RNNs:} The project team has demonstrated significant success in efficiently evolving recurrent neural networks with multiple memory cell types for time-series data prediction problems~\cite{desell-exalt-coal-2019,desell-gecco-examm-2019}. To rapidly adjust to new sensors and minimal amounts of data, we will develop an evolutionary strategy that takes RNNs pre-trained on similar/related data, and evolves them in parallel using new sensor data as it arrives. This will bootstrap the learning process to allow accurate data and confidence predictions using new data.

\vspace{1mm} \noindent \textbf{Objective III: Adaptive Control Loop Integration:} We will demonstrate TVA's ability to positively impact the decision-making process in autonomous systems by integrating it into the popular MAPE-K~\cite{kephart2003vision} and Rainbow~\cite{garlan2004rainbow} self-adaptive control loops. %DK: Add more to this? <-- ALEX: I think this is fine for now, unless you want to add a sentence explicitly proposing one way you would want to actually integrate the items above into MAPE-K/Rainbow....

\vspace{1mm} \noindent \textbf{Objective IV: Creation of Dataset Containing Tactic Volatility:} While there are several available datasets that have been widely used~\cite{SEAMS_Artifacts_URL, traffic_Archive_URL}, there exists no significantly-sized dataset that contains time-series data with real-world tactic volatility. %, specifically in the form of cost and latency. %This lack of existing, easy to adopt, tactic volatility data not only makes the evaluation of potential decision-making mechanisms much more difficult, but can also lead to inaccurate findings due to the `one-off' nature of self-created simulation tools and datasets. 
In our work, we will create a time-series dataset containing real world, tactic volatility using physical devices such as UAVs and robotic fish, furthermore building on our prior work in tactic volatility simulations and in creating such datasets \cite{VALET_TOOL_URL}. 
%This will be accomplished by collecting time series data from physical robotic devices such as UAVs and robotic fish. %APIs will be used to collect the time series data (including cost, latency, \etc) from these devices. 
%Furthermore, we will build on our prior work in generating publicly available datasets that emulate real-world tactic volatility~\cite{VALET_TOOL_URL} (offering simulation tools to enable rapid prototyping). % <-- ALEX: i reworded this last sentence, b/c it initially contradicted the sentence before by saying we have already achieved the objective of creating a volatility dataset

%There are no known significantly sized, existing datasets that contain real-world volatility. This limits the ability to evaluate processes and their ability to account for real-world variability. We will create a large, publicly available 


%MAPE-K~\cite{kephart2003vision} is a popular adaptation loop used in a large number of autonomous systems. We will integrate our TVA process into MAPE-K to demonstrate its effectiveness, and ability to enhance the decision-making process in a large number of autonomous systems.

%\newpage % for appearances
\vspace{1mm} \noindent \textbf{Objective V: Component and System Evaluation:} A systematic evaluation will be conducted at both the component and system levels to evaluate the overall effectiveness of our framework in supporting the prediction of tactic volatility in autonomous systems. Existing simulation tools such as SWIM~\cite{moreno2018swim} and DARTSIM~\cite{MorenoDART2019} will be used to demonstrate the positive impact of TVA in terms of system efficiency, resiliency, and ability to complete mission-critical operations.

%We will establish the positive impact that our TVA process will have on a system's efficiency, resiliency and ability to complete system/mission critical operations using our created dataset.

\vspace{-3mm}
\subsection{Proposed Tactic Volatility Aware (TVA) Technique}

%Our TVA process consists of two phases: (I) Forecasting SLA specifications using \emph{Autoregressive Integrated Moving Average} (ARIMA) (II) Estimating tactic latency and cost using Multiple Regression Analysis (MRA).\dan{update these steps}

The primary phases of our proposed TVA process are:
%Our TVA process will consist of several primary phases which are described below:
\vspace{-1mm}
\begin{enumerate}[noitemsep]

\item \textbf{Tactic Volatility Prediciton: } We will adapt our proposed Evolutionary Exploration of Augmenting Memory Models (EXAMM) algorithm for evolving RNNs that process streaming time series data from system sensors.  We will extend EXAMM to utilize a more robust, complex objective function suitable for streaming sample streams and, furthermore, incorporate a measure of confidence -- the RNNs we evolve/adapt will provide running variance/standard deviation estimates alongside their primary mean predictions of future values.

\item \textbf{Bootstrapping Trained RNNs to New Scenarios and Data:} We will investigate and develop mechanisms for using previously-trained RNNs and evolving them to quickly predict new parameters with limited available data (a sort of dynamic transfer learning). This will be useful, for example, if a new sensor system becomes available or when a network needs to be re-evolved if a sensor(s) is damaged or becomes unavailable.
%\dan{we should add in a few words how this fits into TVA better.}

\item \textbf{Incorporate predictions into the decision-making process:} The estimated tactic values will be integrated into autonomous systems to facilitate vastly improved decision-making.
%This will enable it to make better-informed decisions leading to more optimal outcomes.



%The anticipated tactic volatility values will be incorporated into the decision-making processes in several ways. The estimated tactic values will be included in the system's \emph{utility} equation which is often used to guide the system's actions and the decisions-made by the system. The estimated values will also be \hl{compared} with the specifications defined in the SLA to enable the system to act more proactively when necessary to ensure that the specifications defined in the SLA are adhered to.\dan{rewrite this}

%% DK: I am not sure that this step is actually needed since it is kind of explained in the previous step
%\item \textbf{Perform system operations using predictions: }The input values will enable the system to make more appropriate and informed actions, thus supporting it in dynamic and volatile environments. \dan{rewrite this}
	
\end{enumerate}



%%%%%%%%%%%%%%%%%%%%%%%%%%%%%%%%%%%%%%%%%%%%%%%%%%%%%%%%%%%%%%

%DK: Take this out if space is an issue since it may be confusing to people
%A high level overview of TVA is shown in Figure \ref{fig:TVA_workflow}.


% DK: Maybe remove the tactic observations part

%\tikzstyle{process} = [rectangle,rounded corners, text centered, draw=black, minimum width=.5cm, minimum height=1.25cm,text width=63.3]

%\tikzstyle{arrow} = [thick,->,>=stealth]

%\begin{figure}[h]
%\centering
%\begin{tikzpicture}[node distance=0.2cm]

%\node (a) [process, xshift=-.0cm] {Tactic Observations};
%\node (b) [process, below of=a, node distance=2.0cm] {xxPrediction XXX};


%\node (c) [process, right of=a, node distance=4.5cm] {SLA};
%\node (d) [process, below of=c, node distance=2cm] {Decision Process};

%%\node (e) [process, below right of=c, xshift=2.3cm, node distance=1.5cm] {System Actions};

%\node (e) [process, right of=d, xshift=2.3cm, node distance=1.5cm] {System Actions};

%\node[draw, thick, dotted, rounded corners, inner xsep=1em, inner ysep=1em, fit=(a) (b)] (box) {};

%\node[fill=white] at (box.north) {Text Here};


%\node[draw, thick, dotted, rounded corners, inner xsep=1em, inner ysep=1em, fit=(c) (d)] (box) {};

%\node[fill=white] at (box.north) {Text Here};


% \draw[black,thick,dotted] ($(J.north west)+(-1.6,1.0)$)  rectangle ($(L.south east)+(1.6,-3.0)$);
 
%  \draw[red,thick,dotted] ($(J.north west)+(2.0,1.0)$)  rectangle ($(L.south east)+(5.0,-3.0)$);

%\node (a) [process, xshift=-.4cm] {Tactic Observations};
%\node (b) [process, right of=a, xshift=2.9cm] {BRR Prediction};
%\node (c) [process, right of=b, xshift=2.9cm] {Utility Calculation};
%\node (d) [process, right of=c, xshift=2.9cm] {Tactic Planning};
%\node (a) [process, xshift=-.4cm] {CCCC};
%\node (a) [process, xshift=-.4cm] {DDDDD};



%\node (sa) [process, xshift=2.5cm] {BRR Latency Prediction};
%\node (analysis) [process, right of=sa, xshift=2.7cm]  {Utility Equation/Select Tactic};
%\node (analysis) [process, right of=sa, xshift=2.7cm]  {Execute Tactic};


% \draw [arrow] (a) -- (b);
% \draw [arrow] (b) -- (d);
% \draw [arrow] (c) -- (d);
% \draw [arrow, ultra thick] (d) -- (e);
 

%\end{tikzpicture}
%\caption{Basic Workflow of TVA\dan{This needs to be modified/expanded}} 
%\label{fig:TVA_workflow}
%\end{figure}


\vspace{-8mm}
\section{Future Naval Relevance}
\vspace{-2mm}
% Future Naval Relevance – A description of potential Naval relevance and contributions of the effort to the agency’s specific mission.


The proposed work closely adheres to the goals of the \emph{Naval Research and Development Framework} in the categories of `Augmented Warfighter' and `Integrated \& Distributed Forces'. This research will directly benefit innumerable autonomous processes ranging from small cyber-physical systems to large UAVs, making them more effective and resilient. To support autonomous systems in real-world settings, a tactic volatility aware solution is needed to increase the system's efficiency, predictability, resiliency, and ability to complete mission and system critical operations. 

\vspace{-5mm}
\section{Navy Lead}
\vspace{-6mm}
% Navy Lead: Provide the name, organizational code, and phone number of the NAWCAD lead, if known. Leave this entry blank if there is no Navy Lead.




% DK: Maybe just leave these as uknown
%name, organizational code, and phone number

%%%%%%%%%%%%%%%
    \begin{multicols}{2}
    \begin{itemize}[noitemsep]
        \item \textbf{Name: }Dave Gerkin 
        \item \textbf{Organizational code: }\emph{unknown}%\hl{XXXX}\dan{add}   %  Engineering Education and Research Partnerships at Naval Air Warfare Center
        \item \textbf{Email: }David.Gerkin@Navy.mil
        \item \textbf{Phone Number: }(301)-342-8323

    \end{itemize}
    \end{multicols}


%%%%%%%%%%%%%%%%%%%%%%

\vspace{-8mm}
\section{Cost Estimate} % Required Section
\vspace{-2mm}
% Double check this cost number

% DK ? - Mention something about the ONR project?

The total budget is \$150k/yr for a total cost of \$450k for the three year project. Our budget will include summer support for the project team  II) PhD Student support III) Equipment (IV) Travel for dissemination. The project team has experience successfully executing DOD projects. %and have collectively authored more than \hl{XXX} works, many of which have appeared in top-tier venues. 
PI Krutz was a 2018 Research Fellow for the AFRL working in the area of artificial intelligence. Project members include: %A broad overview of our funding plan is shown in Table~\ref{table:budget}.


%\begin{table}[h]
%\begin{center}
%\caption{Budget Summary}
%\label{table:budget}

%\begin{tabular}{ c | c | c  }
			
%  \textbf{Y1} & \textbf{Y2} & \textbf{Y3} \\   \hline
%  175k & 175k & 175k \\
%   \end{tabular}
%  \end{center}
%\end{table}


%   ? What is the typical budget
% 1 PhD student supervised by Krutz/Qi, the other by Travis/Alex (2 total)

%\vspace{-5mm}
% Not sure we should have this section
%\subsection*{Project Team}

\vspace{-3mm}
%\setlength\columnsep{65pt}
 \begin{multicols}{2}
     \begin{itemize}[noitemsep, leftmargin=*]
        \item Daniel Krutz - \emph{Autonomous Systems}
        \item Qi Yu- \emph{Dynamic Models, Bayesian Learning}
        \item Travis Desell - \emph{Time Series Prediction, RNNs}
        \item Alex Ororbia - \emph{Lifelong machine learning}
%        \item item 5
%        \item item 6
    \end{itemize}
\end{multicols}



%%%%%%%%%%%%%%%%%%%%%%%%%%%%%%%%

\label{lastpage}
\cleardoublepage


\appendix


%% DK: Put onto a different page since it does not count against the page limit
\setcounter{page}{1}

\cfoot{\thepage}
\pagenumbering{roman}
%\section{Appendix}

%\input{sections/Appendix.tex}


%% I think the appendix goes before the bib. Otherwise, I could see people missing it
\newpage
\pagebreak
\addcontentsline{toc}{section}{References}
\bibliographystyle{plain}
\bibliography{refs}

\end{document}
